% Options for packages loaded elsewhere
\PassOptionsToPackage{unicode}{hyperref}
\PassOptionsToPackage{hyphens}{url}
\documentclass[
]{article}
\usepackage{xcolor}
\usepackage{amsmath,amssymb}
\setcounter{secnumdepth}{-\maxdimen} % remove section numbering
\usepackage{iftex}
\ifPDFTeX
  \usepackage[T1]{fontenc}
  \usepackage[utf8]{inputenc}
  \usepackage{textcomp} % provide euro and other symbols
\else % if luatex or xetex
  \usepackage{unicode-math} % this also loads fontspec
  \defaultfontfeatures{Scale=MatchLowercase}
  \defaultfontfeatures[\rmfamily]{Ligatures=TeX,Scale=1}
\fi
\usepackage{lmodern}
\ifPDFTeX\else
  % xetex/luatex font selection
\fi
% Use upquote if available, for straight quotes in verbatim environments
\IfFileExists{upquote.sty}{\usepackage{upquote}}{}
\IfFileExists{microtype.sty}{% use microtype if available
  \usepackage[]{microtype}
  \UseMicrotypeSet[protrusion]{basicmath} % disable protrusion for tt fonts
}{}
\makeatletter
\@ifundefined{KOMAClassName}{% if non-KOMA class
  \IfFileExists{parskip.sty}{%
    \usepackage{parskip}
  }{% else
    \setlength{\parindent}{0pt}
    \setlength{\parskip}{6pt plus 2pt minus 1pt}}
}{% if KOMA class
  \KOMAoptions{parskip=half}}
\makeatother
\setlength{\emergencystretch}{3em} % prevent overfull lines
\providecommand{\tightlist}{%
  \setlength{\itemsep}{0pt}\setlength{\parskip}{0pt}}
\usepackage{bookmark}
\IfFileExists{xurl.sty}{\usepackage{xurl}}{} % add URL line breaks if available
\urlstyle{same}
\hypersetup{
  hidelinks,
  pdfcreator={LaTeX via pandoc}}

\author{}
\date{}

\begin{document}

\section{INSTITUT UNIVERSITAIRE DES SCIENCES -
IUS}\label{institut-universitaire-des-sciences---ius}

\subsection{\texorpdfstring{\textbf{Faculté des Sciences et Technologie
-
FST}}{Faculté des Sciences et Technologie - FST}}\label{facultuxe9-des-sciences-et-technologie---fst}

Niveau L3 Sciences Informatiques

\textbf{NOM} : FABIEN

\textbf{Prenom} : Marie Béatrice

Soumis au chargé de cours Ismael SAINT AMOUR

\textbf{Date}: Dimanche 23 Mars 2025

\section{Découvrir comment installer et intégrer GNS3 avec Vmware
Workstation pour utiliser une machine
virtuelle}\label{duxe9couvrir-comment-installer-et-intuxe9grer-gns3-avec-vmware-workstation-pour-utiliser-une-machine-virtuelle}

\subsubsection{- comme serveur dans un environnement de simulation
réseau. Installer et configurer GNS3 sur votre machine. Installer
VirtualBox et créer une machine virtuelle qui jouera le rôle de
serveur.Importer la VM VirtualBox dans GNS3 et l’intégrer dans une
topologie réseau simulée.Découvrir l’outil GNS3 pour la simulation de
réseaux.}\label{comme-serveur-dans-un-environnement-de-simulation-ruxe9seau.-installer-et-configurer-gns3-sur-votre-machine.-installer-virtualbox-et-cruxe9er-une-machine-virtuelle-qui-jouera-le-ruxf4le-de-serveur.importer-la-vm-virtualbox-dans-gns3-et-lintuxe9grer-dans-une-topologie-ruxe9seau-simuluxe9e.duxe9couvrir-loutil-gns3-pour-la-simulation-de-ruxe9seaux.}

\subsection{Installation de GNS3 et Importation de GNS3 dans
VirtualBox}\label{installation-de-gns3-et-importation-de-gns3-dans-virtualbox}

\begin{figure}
\centering
{image 1}
\caption{image 1}
\end{figure}

\begin{figure}
\centering
{image 2}
\caption{image 2}
\end{figure}

\begin{figure}
\centering
{image 3}
\caption{image 3}
\end{figure}

\begin{figure}
\centering
{image 3}
\caption{image 3}
\end{figure}

\begin{figure}
\centering
{image 3}
\caption{image 3}
\end{figure}

\begin{figure}
\centering
{image 3}
\caption{image 3}
\end{figure}

\subsection{Ajout d’une image de
Routeur}\label{ajout-dune-image-de-routeur}

\begin{figure}
\centering
{image 4}
\caption{image 4}
\end{figure}

\begin{figure}
\centering
{image 3}
\caption{image 3}
\end{figure}

\begin{figure}
\centering
{image5}
\caption{image5}
\end{figure}

\section{Topologie du réseau}\label{topologie-du-ruxe9seau}

\textbf{Connectez le routeur au switch. Connectez les PC au switch.}

\begin{figure}
\centering
{image6}
\caption{image6}
\end{figure}

\textbf{Configuration du Routeur et des deux PCS}

\begin{figure}
\centering
{image7}
\caption{image7}
\end{figure}

\begin{figure}
\centering
{image8}
\caption{image8}
\end{figure}

\begin{figure}
\centering
{image9}
\caption{image9}
\end{figure}

\textbf{Reproduisez cette topologie en configurant le routeur et les
PC.}

\begin{figure}
\centering
{image10}
\caption{image10}
\end{figure}

\begin{figure}
\centering
{image11}
\caption{image11}
\end{figure}

\section{Conclusion}\label{conclusion}

Ce TD m’aide à installer et intégrer GNS3 pour utiliser une machine
virtuelle comme serveur dans un environnement de simulation réseau.

\end{document}
